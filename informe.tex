\documentclass{article}
\usepackage{amsfonts}          % Para las negrita de pizarra
\usepackage{indentfirst}       % Para que quede mas lindo el formateo
\usepackage{graphicx}          % Para graficos
\usepackage{minted}            % Para poner codigo y que quede con sintaxis fachera
\usepackage{hyperref}          % Para meter hipervinculos
\usepackage[dvipsnames]{xcolor}% Para usar colores
\usepackage{hhline}            % Mas configuracion para las líneas en tablas
\usepackage{amsmath}           % Agregado para tags de ecuaciones
\usepackage{xcolor}            % Coloreado de ecuaciones
\usepackage{quoting, xparse}   % Usado para citar
% \usepackage{svg}               % Para usar imagenes svg que se ven lindas independientemente del zoom. WARNING REQUIERE DE INKSCAPE. Tal vez no vale la pena
\usepackage{amsmath}

\graphicspath{ {./images/} }

\newcommand{\docuPy}{%
  {\href{https://wiki.python.org/moin/TimeComplexity}{documentacion oficial}}
  }%

  % Comandos para facilitar el citado
  % Fuente: https://tex.stackexchange.com/a/391739/273865
\NewDocumentCommand{\bywhom}{m}{% the Bourbaki trick
  {\nobreak\hfill\penalty50\hskip1em\null\nobreak
   \hfill\mbox{\normalfont(#1)}%
   \parfillskip=0pt \finalhyphendemerits=0 \par}%
}

\begin{document}

\begin{titlepage}
  \vspace*{1cm}

  \begin{center}
    {\Huge{Trabajo Práctico 3: Problemas NP-Completos para la defensa de la Tribu del Agua}}
  \end{center}

  \vspace{0.4cm}

  \begin{center}
    {\LARGE{Facultad de Ingeniería de la Universidad de Buenos Aires}}\\
    \vspace{0.3cm}
    {\Large{Teoría de Algoritmos}}\\
    \vspace{0.3cm}
    {\large{Cátedra Buchwald-Genender}}\\
  \end{center}

  \vspace{0.8cm}
  \begin{center}
    \includegraphics[scale=0.8]{Logo-fiuba}
  \end{center}

  \vspace{1.4cm}
  \begin{center}

    {\begin{minipage}[t]{.32\textwidth}
        \begin{center}
          Gómez Belis, Sofía\\
          {\small{Padrón: 109358}}\\
          {\small{email: sgomezb@fi.uba.ar}}
        \end{center}
          \end{minipage}
          \begin{minipage}[t]{.32\textwidth}
        \begin{center}
          Llanos Pontaut, Valentina\\
          {\small{Padrón: 104413}}\\
          {\small{email: vllanos@fi.uba.ar}}\\
        \end{center}
      \end{minipage}
      \begin{minipage}[t]{.32\textwidth}
        \begin{center}
          Orsi, Tomas Fabrizio\\
          {\small{Padrón: 109735}}\\
          {\small{email: torsi@fi.uba.ar}}
        \end{center}
      \end{minipage}}

  \end{center}
\end{titlepage}

\renewcommand*\contentsname{Indice}
\tableofcontents
\pagebreak

\section{Introducción}
\subsection{Descripción y objetivo}

Continuando con el ataque de la Nación del Fuego sobre el resto de las naciones, esta vez es la Tribu del Agua la que requiere de nuestra ayuda para defenderse. 

Cada maestro agua tiene una fuerza o habilidad positiva $x_i$, y contamos con el conjunto de todos los valores $(x_i, x_2, \dots, x_n)$. Basándonos en estos, el maestro Pakku desea separar los maestros en \texttt{k} grupos $(S_1, S_2, \dots, S_k)$ parejos tal que cuando un grupo se canse, entrará el siguiente en el combate, obteniendo un ataque constante que les permita salir victoriosos, aprovechando también la ventaja del agua por sobre el fuego.

Para que los grupos estén lo más parejos posibles, nos han encomendado minimizar la adición de los cuadrados de las sumas de las fuerzas de los grupos:

$$
\min \sum_{i=1}^{k} \left( \sum_{x_j \in S_i} x_j \right)^2
$$

En este trabajo desarrollaremos algoritmos de backtracking, programación lineal y posibles aproximaciones buscando resolver el problema planteado con el objetivo de ayudar a los mestros de la Tribu del Agua a derrotar a la Nación del Fuego.

\section{Demostración de problema NP-Completo}
\subsection{Problema NP}
\subsection{Reducción}

\section{Complejidad algorítmica}
\subsection{Complejidad lectura de archivos}

\subsection{Complejidad algoritmo de Backtracking}
\subsection{Complejidad algoritmo de programación lineal}
\subsection{Complejidad algoritmo de aproximación}

\subsection{Efecto de las variables sobre el algoritmo}

\section{Ejemplos de ejecución}
\label{sec:ejemplos}
\section{Mediciones de tiempo}
\label{sec:medTiempo}
\subsection{Algoritmo de backtracking}
\subsection{Algoritmo de programación lineal}
\subsection{Algoritmo de aproximación}


\section{Conclusión}


\end{document}
